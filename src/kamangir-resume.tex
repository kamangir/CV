% Arash Kamangir, CV
% Revision 1.03 - Maintained by Arash Abadpour

\documentclass[12pt,a4paper]{moderncv}
\moderncvtheme[grey]{casual}%classical/casual, roman, blue/orange/red/green/grey
\usepackage[utf8]{inputenc}
\usepackage[scale=0.8]{geometry}
%\setlength{\hintscolumnwidth}{3cm}
% if you want to change the width of the column with the dates
%\AtBeginDocument{\setlength{\maketitlenamewidth}{6cm}}  % only for the classic theme, if you want to change the width of your name placeholder (to leave more space for your address details
\AtBeginDocument{\recomputelengths} % required when changes are made to page layout lengths

% personal data
\firstname{Arash}
\familyname{Abadpour}
\title{Resume}
\mobile{+1 647 567 3487}
\email{arash.abadpour@gmail.com}
\extrainfo{http://abadpour.com}

\begin{document}
\maketitle

\section{ Biography}
\cvline{}{Arash Abadpour left Iran in September 2005 to continue his studies in Canada. He has a Ph.D. in Electrical and Computer Engineering and works as a professional researcher in the field of Image Processing and Machine Vision. }

\section{Blogging}
\cvline{}{Arash started blogging in October 2004 in English under the pen name Kamangir (Archer). He started a Persian blog with the same name in May 2007. He now blogs at \href{http://kamangir.net}{kamangir.net}. English Kamangir was deactivated in 2012.}
\cvline{}{Based on different independent statistics, collected in 2012, Persian Kamangir is among the twenty most-read Persian blogs on the net. Posts from Persian Kamangir are widely shared on different social networks and are referenced by other Persian bloggers as well as Persian media sources. Several Persian-language news and media websites have reproduced or actively reproduce content from Persian Kamangir in their websites. }

\section{ Media}
\cvline{}{Arash acts as a commentator for the media, both in English and in Persian, on the Persian blogosphere and its correspondence with the sociopolitical situation in Iran. He has been interviewed by Persian BBC, Persian DW-World, Aljazeera, Radio Farda, Voice of America, Washington Post, ZDF, and Radio France Internationale, among others. }

\section{Writing}
\cvline{}{Arash writes, both in Persian and in English, about the Persian blogosphere. His writings have appeared in Persian BBC, DW-World, Article 19, and Pajamas Media, among others. }

\section{Panels and Presentations}
\cvline{}{Arash has been on panels and have presented on the use of the Internet in Iran on different occasions in conferences held in Canada, USA, Germany, and Denmark, among others. }

\newpage
\section{Community Building}
\cvline{}{Arash maintains active contact with the body of the Persian blogosphere located in Iran. He maintains active accounts on Facebook, Google+, Twitter, and other social networks, where he is among the most subscribed-to users. He has facilitated contact with Iranian bloggers as requested by the media and researchers in the field. He also facilitates publication of content produced inside Iran in media sources located outside the country. Arash provides technical and logistical support for Persian bloggers based in Iran and has been actively involved in several social campaigns.}
\cvline{}{Arash was the Persian-speaking jury member for the DW-World Best of Blogs international blogs awards in 2011, 2012, and 2013.}

\section{Human Rights Activism}
\cvline{}{Arash actively supports the non-violent pro-human rights movement on the web through writing blog posts, signing open letters, and encouraging activism through different social networks. }

\section{Training}
\cvline{}{Arash has provided training sessions and materials on the topics of data journalism, data visualization, effective content marketing on the web, use of social networks for social/political campaigning, and online security, in programs offered by Iran Media Program and Mianeh School of Journalism, among others.}

\section{Research}
\cvline{}{Arash performs research, both as an independent researcher and as a team member, on issues related to the use of the Internet in Iran. %He established the social media consultant corporation Kamangir Inc. in Ontario, Canada, in October 2012. Kamangir Inc. specializes in providing access to Persian bloggers and Iranian users on the social media as well as performing research and training. 
% Detailed list of publications available on \href{http://abadpour.com}{abadpour.com}.}

\section{Translation Services}
\cvline{}{Arash has worked with several different organizations on projects related to the translation of content from Persian to English and in reverse. Among the organizations Arash has worked with are the Electronic Frontier Foundation (EFF), and Reporters Without Borders (RSF). Arash specializes in translation of Persian blog content into English as well as translation of surveys and questionnaires and technical documents from English to Persian. Arash has been involved in the translation of tutorials, documentations, and user interfaces for different filtering circumvention/assessment tools from English to Persian.}

%\newpage
%\section{More Information}
%\cvline{}{More details about the activities of Arash Abadpour can be found on \href%{http://abadpour.com}{abadpour.com}. Specifics on projects are available upon request and depending on sensitivity concerns.}


\vspace{0.5cm}

Revision:\space1.05

Build Date:\space\today

\end{document}